%	-------------------------------------------------------------------------------
%
%		2020년 7월 12일 첫 작업
%
%
%
%
%
%
%	-------------------------------------------------------------------------------

%\documentclass[10pt,xcolor=pdftex,dvipsnames,table]{beamer}
%\documentclass[10pt,blue,xcolor=pdftex,dvipsnames,table,handout]{beamer}
%\documentclass[14pt,blue,xcolor=pdftex,dvipsnames,table,handout]{beamer}
\documentclass[aspectratio=1610,17pt,xcolor=pdftex,dvipsnames,table,handout]{beamer}

		% Font Size
		%	default font size : 11 pt
		%	8,9,10,11,12,14,17,20
		%
		% 	put frame titles 
		% 		1) 	slideatop
		%		2) 	slide centered
		%
		%	navigation bar
		% 		1)	compress
		%		2)	uncompressed
		%
		%	Color
		%		1) blue
		%		2) red
		%		3) brown
		%		4) black and white	
		%
		%	Output
		%		1)  	[default]	
		%		2)	[handout]		for PDF handouts
		%		3) 	[trans]		for PDF transparency
		%		4)	[notes=hide/show/only]

		%	Text and Math Font
		% 		1)	[sans]
		% 		2)	[sefif]
		%		3) 	[mathsans]
		%		4)	[mathserif]


		%	---------------------------------------------------------	
		%	슬라이드 크기 설정 ( 128mm X 96mm )
		%	---------------------------------------------------------	
%			\setbeamersize{text margin left=2mm}
%			\setbeamersize{text margin right=2mm}

	%	========================================================== 	Package
		\usepackage{kotex}						% 한글 사용
		\usepackage{amssymb,amsfonts,amsmath}	% 수학 수식 사용
		\usepackage{color}					%
		\usepackage{colortbl}					%


	%		========================================================= 	note 옵션인 
	%			\setbeameroption{show only notes}
		

	%		========================================================= 	Theme

		%	---------------------------------------------------------	
		%	전체 테마
		%	---------------------------------------------------------	
		%	테마 명명의 관례 : 도시 이름
%			\usetheme{default}			%
%			\usetheme{Madrid}    		%
%			\usetheme{CambridgeUS}    	% -red, no navigation bar
%			\usetheme{Antibes}			% -blueish, tree-like navigation bar

		%	----------------- table of contents in sidebar
			\usetheme{Berkeley}		% -blueish, table of contents in sidebar
									% 개인적으로 마음에 듬

%			\usetheme{Marburg}			% - sidebar on the right
%			\usetheme{Hannover}		% 왼쪽에 마크
%			\usetheme{Berlin}			% - navigation bar in the headline
%			\usetheme{Szeged}			% - navigation bar in the headline, horizontal lines
%			\usetheme{Malmoe}			% - section/subsection in the headline

%			\usetheme{Singapore}
%			\usetheme{Amsterdam}

		%	---------------------------------------------------------	
		%	색 테마
		%	---------------------------------------------------------	
%			\usecolortheme{albatross}	% 바탕 파란
%			\usecolortheme{crane}		% 바탕 흰색
%			\usecolortheme{beetle}		% 바탕 회색
%			\usecolortheme{dove}		% 전체적으로 흰색
%			\usecolortheme{fly}		% 전체적으로 회색
%			\usecolortheme{seagull}	% 휜색
%			\usecolortheme{wolverine}	& 제목이 노란색
%			\usecolortheme{beaver}

		%	---------------------------------------------------------	
		%	Inner Color Theme 			내부 색 테마 ( 블록의 색 )
		%	---------------------------------------------------------	

%			\usecolortheme{rose}		% 흰색
%			\usecolortheme{lily}		% 색 안 칠한다
%			\usecolortheme{orchid} 	% 진하게

		%	---------------------------------------------------------	
		%	Outter Color Theme 		외부 색 테마 ( 머리말, 고리말, 사이드바 )
		%	---------------------------------------------------------	

%			\usecolortheme{whale}		% 진하다
%			\usecolortheme{dolphin}	% 중간
%			\usecolortheme{seahorse}	% 연하다

		%	---------------------------------------------------------	
		%	Font Theme 				폰트 테마
		%	---------------------------------------------------------	
%			\usfonttheme{default}		
			\usefonttheme{serif}			
%			\usefonttheme{structurebold}			
%			\usefonttheme{structureitalicserif}			
%			\usefonttheme{structuresmallcapsserif}			



		%	---------------------------------------------------------	
		%	Inner Theme 				
		%	---------------------------------------------------------	

%			\useinnertheme{default}
			\useinnertheme{circles}		% 원문자			
%			\useinnertheme{rectangles}		% 사각문자			
%			\useinnertheme{rounded}			% 깨어짐
%			\useinnertheme{inmargin}			




		%	---------------------------------------------------------	
		%	이동 단추 삭제
		%	---------------------------------------------------------	
%			\setbeamertemplate{navigation symbols}{}

		%	---------------------------------------------------------	
		%	문서 정보 표시 꼬리말 적용
		%	---------------------------------------------------------	
%			\useoutertheme{infolines}


			
	%	---------------------------------------------------------- 	배경이미지 지정
%			\pgfdeclareimage[width=\paperwidth,height=\paperheight]{bgimage}{./fig/Chrysanthemum.jpg}
%			\setbeamertemplate{background canvas}{\pgfuseimage{bgimage}}

		%	---------------------------------------------------------	
		% 	본문 글꼴색 지정
		%	---------------------------------------------------------	
%			\setbeamercolor{normal text}{fg=purple}
%			\setbeamercolor{normal text}{fg=red!80}	% 숫자는 투명도 표시


		%	---------------------------------------------------------	
		%	itemize 모양 설정
		%	---------------------------------------------------------	
%			\setbeamertemplate{items}[ball]
%			\setbeamertemplate{items}[circle]
%			\setbeamertemplate{items}[rectangle]






		\setbeamercovered{dynamic}





		% --------------------------------- 	문서 기본 사항 설정
		\setcounter{secnumdepth}{5} 		% 문단 번호 깊이
		\setcounter{tocdepth}{5} 			% 문단 번호 깊이




% ------------------------------------------------------------------------------
% Begin document (Content goes below)
% ------------------------------------------------------------------------------
	\begin{document}
	

			\title{찬 불 가}

			\author{김대희}

			\date{2020년 07월 12일}


	%	==========================================================
	%		개정 이력
	%	----------------------------------------------------------
	%		2020.07.12 첫 작성
	%	----------------------------------------------------------
	%	
	%	----------------------------------------------------------


	%	==========================================================
	%
	%	----------------------------------------------------------
		\begin{frame}[plain]
		\titlepage
		\end{frame}



%		\begin{frame} [plain]{목차}
		\begin{frame} {목차}
		\tableofcontents
		\end{frame}
		

	%	========================================================== 	개요
	%		Frame
	%	----------------------------------------------------------
		\part{개요}
		\frame{\partpage}


		\begin{frame} [plain]{목차}
		\tableofcontents
		\end{frame}
		

		
				
		
	%	 ---------------------------------------------------------- 삼귀의 }
	%	 Frame
	%	 ----------------------------------------------------------
		\section{ 삼귀의 }
%		\frame [plain] {\sectionpage}
		

		\begin{frame} [t,plain]
			\begin{block} { 삼귀의 }

			\setlength{\leftmargini}{2em}			
			\begin{itemize}
				\item 작사
				\item 작곡 최영철

				\item 거룩한 부처님께 귀의합니다.
				\item 거룩한 가르침에 귀의합니다.
				\item 거룩한 스님들께 귀의합니다.
			\end{itemize}
			
			\end{block}
		\end{frame} 	%	 --------------------------------------------

		

	%	 ---------------------------------------------------------- 보현행원 }
	%	 Frame
	%	 ----------------------------------------------------------
		\section{ 보현행원 }
%		\frame [plain] {\sectionpage}
		

		\begin{frame} [t,plain]
			\begin{block} { 보현행원 1}

			\setlength{\leftmargini}{2em}			
			\begin{itemize}
				\item 작사
				\item 작곡
				\item . 내 이제 두 손-모아 청하옵나-니
시방세계 부처-님 우주 대-광-명
두 눈 어둔 이내몸 굽어 살피-사
위없는 대법-문을 널리 여-소-서

				\item 허공계와 중생-계가 다 할지라-도
오늘-세운 이-서원은 끝없아-오-리

			\end{itemize}
			
			\end{block}

		\end{frame} 	%	 --------------------------------------------


		\begin{frame} [t,plain]
			\begin{block} { 보현행원 2}

			\setlength{\leftmargini}{2em}			
			\begin{itemize}
				\item  내 이제 엎드-려서 원하옵나-니
영겁토록 열반-에 들지 맙-시-고
이 세상에 중생을 굽어 살피-사
삼계화택 심한-고난 구원 하-소-서

				\item 허공계와 중생-계가 다 할지라-도
오늘-세운 이-서원은 끝없아-오-리
			\end{itemize}
			
			\end{block}

		\end{frame} 	%	 --------------------------------------------
		

	%	 ---------------------------------------------------------- 청법가 }
	%	 Frame
	%	 ----------------------------------------------------------
		\section{ 청법가 }
%		\frame [plain] {\sectionpage}
		

		\begin{frame} [t,plain]
			\begin{block} { 청법가 }

			\setlength{\leftmargini}{2em}			
			\begin{itemize}
				\item 작사  이광수 정운문
				\item 작곡  이찬우

				\item 덕높-으신 스-승님 사자-좌에 오르사-
				\item    사자-후를 합-소서 감로-법을 주-소서
				\item    옛인연을 이어서 새인연을 맺-도록
				\item    대자-비를 베-푸사 법을- 설하옵-소서
			\end{itemize}
			
			\end{block}
		\end{frame} 	%	 --------------------------------------------




	%	 ---------------------------------------------------------- 사홍서원 }
	%	 Frame
	%	 ----------------------------------------------------------
		\section{ 사홍서원 }
%		\frame [plain] {\sectionpage}
		

		\begin{frame} [t,plain]
			\begin{block} { 사홍서원 }

			\setlength{\leftmargini}{2em}			
			\begin{itemize}
				\item 작사
				\item 작곡  최영철

				\item   중생을 다 건지오리다
				\item    번뇌를 다 끊으오리다
				\item    법문을 다 배우오리다
				\item    불도를 다 이루오리다
			\end{itemize}
			
			\end{block}
		\end{frame} 	%	 --------------------------------------------

“중생무변서원도(衆生無邊誓願度) 
번뇌무진서원단(煩惱無盡誓願斷) 
법문무량서원학(法門無量誓願學) 
불도무상서원성(佛道無上誓願成)”이다.

\newpage

	%	========================================================== 함께 읽어볼 자료
		\part{ 함께 읽어볼 자료 }
		\frame{\partpage}
		
		\begin{frame} [plain]{목차}
		\tableofcontents%
		\end{frame}




	%	---------------------------------------------------------- 작곡가 }
	%		Frame
	%	----------------------------------------------------------
		\section{ 작곡가 }
		\begin{frame} [t,plain]
		\frametitle{작곡가 }
			\begin{block} { 작곡가 }
			\setlength{\leftmargini}{4em}			
			\begin{itemize}
				\item 
				\item 
				\item 
				\item 
			\end{itemize}
			\end{block}						
								
		\end{frame} 	%	 --------------------------------------------


	%	---------------------------------------------------------- 작사가 }
	%		Frame
	%	----------------------------------------------------------
		\section{ 작사가 }
		\begin{frame} [t,plain]
		\frametitle{작사가 }
			\begin{block} { 작사가 }
			\setlength{\leftmargini}{4em}			
			\begin{itemize}
				\item 
				\item 
				\item 
				\item 
			\end{itemize}
			\end{block}						
								
		\end{frame} 	%	 --------------------------------------------


	%	---------------------------------------------------------- 합창단 }
	%		Frame
	%	----------------------------------------------------------
		\section{ 합창단 }
		\begin{frame} [t,plain]
		\frametitle{합창단 }
			\begin{block} { 합창단 }
			\setlength{\leftmargini}{4em}			
			\begin{itemize}
				\item 
				\item 
				\item 
				\item 
			\end{itemize}
			\end{block}						
								
		\end{frame} 	%	 --------------------------------------------


% ------------------------------------------------------------------------------ ------------------------------------------------------------------------------ ------------------------------------------------------------------------------
% End document
% ------------------------------------------------------------------------------ ------------------------------------------------------------------------------ ------------------------------------------------------------------------------
\end{document}


	%	----------------------------------------------------------
	%		Frame
	%	----------------------------------------------------------
		\begin{frame} [c]
%		\begin{frame} [b]
%		\begin{frame} [t]
		\frametitle{감리 보고서}
		\end{frame}						

